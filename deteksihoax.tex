\documentclass{article}
\usepackage{enumerate}
\usepackage{multirow}
\begin{document}
\title{ANALISA DAN DETEKSI KONTEN HOAX PADA MEDIA BERITA\\ INDONESIA MENGGUNAKAN MACHINE LEARNING}
\author{Munirul, Ula, Mulia Mahendra Alvanof, Rahmat Triandi}
\maketitle
\begin{center}
Teknik Informatika Universitas Malikussaleh Lhokseumawe

Jl. Cot Tgk Nie-Reulet, Aceh Utara, 141 Indonesia

email : muliamahendraalvanof@gmail.com, rahmattriandi83@gmail.com
\end{center}
\begin{abstract}
\emph{accident}
Sekarang ini konten Hoax yang mengandung informasi tidak benar malah sering kali 
menjadi konsumsi massal pengguna internet. Hal ini merupakan sesuatu yang buruk 
karena dapat meningkatkan rasa tidak percaya terhadap berita dan informasi yang 
ada di internet hingga menimbulkan kebingungan pada masyarakat dalam menentukan 
informasi mana yang benar. 

Dalam Penelitian ini, percobaan yang dilakukan bertujuan untuk memilih algoritma 
terbaik dalam membedakan berita hoax dan berita asli menggunakan metode text 
mining serta pendekatan dengan machine learning dan 150 artikel berbahasa 
Indonesia (50 artikel hoax dan 100 artikel asli) sebagai data yang akan digunakan.

Penelitian ini akan dimulai dengan tahap preprocessing teks yang terdiri dari 
tokenizing, case folding, filtering, stopword removal, stemming dan weighting TF-IDF 
menggunakan penggabungan fitur unigram dan bigram baru kemudian diolah menjadi 
teks klasifikasi. Hasil dari penelitian ini didapatkan kesimpulan bahwa algoritma 
Random Forest memiliki akurasi terbaik dalam mengklasifikasikan berita hoax dan 
berita asli dibandingkan dengan algoritma Multilayer Perceptron, Naïve Bayes,dan 
Support Vector Machine dengan nilai akurasi 75.37%.

\end{abstract}
\section{PENDAHULUAN}
Kemajuan teknologi informasi memiliki berdampak pada semakin maraknya media 
berita online yang dapat diakses dengan mudah. Namun Seiring berkembangnya 
penyebaran berita online, kualitas berita yang disebarkan juga semakin berkurang. 
Berita yang tersebarluas tidak semuanya benar, termasuk didalamnya berita palsu atau 
hoax yang biasanya berakibat merugikan untuk piha tertentu.

Hoax adalah informasi atau berita yang mengandung hal-hal yang tidak pasti atau yang 
tidak berdasarkan fakta atas sesuatu yang benar terjadi. Hoax juga dapat diidentifikasi 
dengan beberapa ciri: berita datang dari sumber yang tidak jelas /tidak dipercaya. 
Gambar, foto atau video digunakan adalah hasil rekayasa, menggunakan kalimat 
provokatif, mengandung politik maupun ras. Penyebaran hoax di kalangan masyarakat 
dapat menyebabkan efek negatif, seperti kerusakan,kerugian, baik materiil dan 
psikologis, hilangnya kepercayaan masyarakat, dan sebagainya. 

Dampak dari penyebaran berita hoax akan memiliki konsekuensi dan bahaya buruk 
untuk banyak pihak, yang mana hoax dapat menyebabkan kerugian dari berbagai 
aspek, baik waktu dan ekonomi, publik panik, memburuknya hubungan sosial dan 
sebagainya. Untuk menghindari dampak buruk ini, penelitian ini akan membantu untuk 
mengklasifikasikan berita sebelum terpengaruh oleh hoax atau bahkan sampai ikut 
menyebarkan hoax. Teknik yang digunakan dalam penelitian ini adalah dengan sistem 
klasifikasi teks menggunakan pendekatan berbasis machine learning. Algoritma yang 
digunakan adalah: Multilayer Perceptron (MLP), Naive Bayes (NB), Random Forest 
(RF), dan Support Vector Machine (SVM).

Penelitian serupa telah dilakukan sebelumnya oleh Rasywir dan Purwarianti[6], 
bedanya antara penelitian ini dan penelitian sebelumnya adalah dalam penggunaan 
algoritma dimana Penelitian Rasywir dan Purwarianti menggunakan tiga 
algoritma,yaitu algoritma Naïve Bayes, Support Vector Machine, dan Decision Tree, 
sementara penelitian ini akan membandingkan empat algoritma yaitu Multilayer 
Perceptron, Naïve Bayes, Support Vector Machine, and Random Forest.

\section{TINJAUAN PUSTAKA}
\subsection{Berita}Berita adalah laporan informasi tentang peristiwa dan pendapat yang aktual, penting 
dan menarik untuk disampaikan kepada publik dalam bentuk surat kabar, radio dan 
media online. Salah satu syarat dari berita adalah bahwa berita tersebut harus 
didasarkan pada keadaan atau peristiwa yang benar-benar terjadi. Tetapi seiring dengan 
kemajuan teknologi yang semakin pesat, penyebaran berita juga semakin tidak jelas 
apakah kebenarannya sesuai dengan fakta atau hanya hoax belaka. Hoax adalah 
manipulasi berita yang disengaja dan bertujuan untuk memberikan informasi atau 
pemahaman yang salah. Hoax sering ditemukan di berita, baik melalui media cetak 
maupun media sosial. Tujuan dari hoax itu sendiri sangat beragam, mulai dari 
menyebarkan ujaran kebencian, menimbulkan kecemasan di masyarakat, memengaruhi 
persepsi masyarakat, dan sebagainya
\subsection{Text Mining}
Text mining adalah salah satu cabang dari data mining yang menganalisis data dalam 
bentuk teks. Text mining itu sendiri adalah suatu proses data mining berupa teks 
dengan sumber data yang biasanya diperoleh melalui dokumen, dan bertujuan untuk 
menemukankata-kata yang dapat mewakili isi dari dokumen.
\subsection{Text Preprocessing}
Text preprocessing adalah tahap awal dari data mining. Pada tahap ini, proses persiapan 
untuk dokumen dan data dilakukan agar dokumen / data siap diproses dan proses 
klasifikasi dapat dilakukan dengan benar. Adapun tahapan dalam Text Preprocessing, 
yaitu:
\begin{enumerate}[{a.}]
\item Tokenizing, 
adalah proses membagi ataumemotong kalimat menjadi kata-kata dengan 
menggunakan spasi, contohnya: "Text preprocessing merupakan tahap awal dari text 
mining ", akan dipecah menjadi: "Text", "preprocessing", "merupakan", "tahap", 
"awal", "dari", "text","mining". 

\item Case folding, 
Case folding adalah tahap di mana semua karakter dalam teks diproses dan diubah 
menjadi huruf kecil. Misalnya: "TEKNOLOGI" akan diubah menjadi "teknologi", 
"Teks" akan dikonversi menjadi kata "teks", dan sebagainya.

\item Filtering,
adalah tahap penghapusan tanda baca

\item Stopwords removing, 
Dalam tahap Stopwords removing kata yang bukan kata unik atau kata-kata yang 
sering muncul namun tidak penting akan dihapus. Contoh kata yang dimaksud adalah 
kata depan, kata sambung, kata keterangan dan kata pengganti, seperti: "yang", "ke", 
"di","sebuah", "pada", "oleh", "ini", "dari", dll.

\item Stemming,
Proses Stemming adalah proses yang dilakukan untuk mendapatkan kata dasar dari 
sebuah kata dengan menghilangkan akhiran dan atau akhiran dari kata tersebut.

\item TF - IDF weighting, 
TF (Term Frequency) adalah frekuensi Munculnya suatu istilah dalam suatu dokumen. 
Semakin besar jumlahkemunculan istilah (TF tinggi) padadokumen, semakin besar 
“bobot” atau nilai kesesuaiannya. IDF (Inverse Document Frequency) adalah 
perhitungan bagaimana istilah tersebut didistribusikan dalam koleksi dokumen terkait. 
IDF menunjukkanhubungan antara ketersediaan istilah dalam semua dokumen. 
\end{enumerate}
\subsection{Tipe Algoritma}
Algoritma yang digunakan dalam penelitian ini meliputi :
\begin{enumerate}[{a.}]
\item Multilayer Perceptron (MLP)
Metode Multilayer Perceptron adalah salah satu topologi artificial neural networks 
yang paling umum digunakan dimana perceptron akan dikoneksikan sehingga 
membuat multiple layers. MLP terdiri atas input, hidden dan output layer dan karena 
terdiri atas beberapa layer, proses perhitungan pada algoritma ini akan melalui beberapa 
tahapan pula.

$$ hidden=\frac{1}{1+exp^ -hiddenNet j}$$

dimana nilai hidden-Netj dapat diambil dari

$$ Hidden_{Netj} = \sum _{i}W_{jk}\times mf_{ij}(Y_{ij})+ bias_{k} $$

Keterangan:

$ W_{jk} $ = "Bobot" neuron j layer sebelumnya ke neuron k pada hidden layer

$ mf_{ij} (Y_{ij}) $ = "Derajat Anggota" atribut i ke kelas j.

$ bias_k $ = "Bias" neuron k.

$$ Output_1 = \frac{1}{1+exp^{-output-Net}k} $$

dimana nilai $ output-Net_k $ dapat diambil dari persamaan

$$ Output_{Netk} = \sum _{j}W_{kl}\times hidden_k + bias_k $$

Keterangan:

$ W_{kl} $ = Bobot" neuron k layer sebelumnya ke neuron l pada hidden layer

$ bias_1 $ = Bias neuron l.

\item Naïve Bayes
Algoritma yang dikembangkan ini telah banyak digunakan sebagai metode untuk
mengklasifikasikan teks yang memiliki "chance" sederhana. Algoritma ini
menggunakan probabilitas dan statistik untuk memprediksi probabilitas masa depan
berdasarkan pengalaman yang ada.

Persamaan teorema Naïve Bayes theorem daat ditulis sebagai berikut :

$$ P(X|Y) = \frac{P(Y|X)\times P(X)}{P(Y)} $$

Keterangan

P(X|Y) = Kemungkinan X Berdasarkan Kondisi Y.

P(Y|X) = Kemungkinan Y Perdasarkan Hipotesis X.

P(X) = Kemungkinan X.

P(Y) = Kemungkinan Y.

\item Support Vector Machine

Metode SVM asalah model universal dari machine learning yang menggunakan 'fungsi
pembatasan linear' sebagai dasar.
Persamaan untuk perhitungan pada algoritma SVM untuk data yang mungkin belum terkelompokan secara benar (Asiyah dan Fithriasari, 2016) adalah:

$$ min\frac{1}{2}||w||^{2}+C\sum ^{\Lambda }_{i=1}\varepsilon i $$

Keterangan:

W = Hyperplane parameter dicari

C = Parameter data error masukan pengguna.

$ \Lambda  $ = Jumlah Partisi/data.

I = Nilai Awal

\item Random Forest

Metode Random Forest adalah pengembangan dari metoder CART, Random Forest
adalah sebuah metode yang dapat meningkatkan akurasi karena metode ini akan
membuat node-node acak untuk setiap node.

$$ Entropy (Y) = -\sum_{i}P(C|Y)\log_{2}P(C|Y) $$ 

Keterangan:

Y = Jumlah kasus.

P(C|Y) = Perbandingan nilaibY terhadap kelas C

\end{enumerate}

\section{METODE PENELITIAN}
\subsection{Sumber Data}
Data yang digunakan dalam penelitian ini diambil dari artikel berita media online yang 
sudah memiliki label hoax dan non-hoax. dengan 50 artikel berlabel hoax dan 100 
artikel berlabel non-hoax. Jadi total keseluruhan artikel yang digunakan sebagai data 
pada penelitian ini adalah 150 artikel.

\subsection{Tahapan Analisa}
Dalam penelitian ini, beberapa tahapan analisa yang digunakan yaitu:
\begin{enumerate}[{a.}]
\item Pengumpulan berita dari situs berita online diantaranya : viva.co.id, detik.com, kompas.com, liputan6.com, metrotvnews.com, 
beritasatu.com, cnnindonesia.com, idntimes.com, republika.co.id, prokal.co, 
cekfakta.com, jpnn.com, okezone.com, sindonews.com, solopos.com, 
tempo.co, merdeka.com, tribunnews.com

\item data berita yang telah dikumpulkan kemudian disimpan dalam format CSV dan kemudian akan melewati tahap text preprocessing, tahap pertama adalah 
tokenizing dimana semua kalimat dari data berita akan dipisahkan 
berdasarkan tanda spasi.

\item Kemudian pada proses folding case, semua huruf diubah menjadi huruf kecil, kemudian kata-kata yang memiliki arti yang sama akan dinormaisasi menjadi 
satu kata yang sama.

\item Pada proses filtering, karakter selain huruf dan angka akan dihapuskan, misalnya tanda baca (.).

\item Pada tahap stopwords removing, kata yang tidak penting dan tidak unik akan  dihapuskan.

\item Selanjutnya pada proses stemming, awalan dan atau akhiran kata dihilangkan, sehingga didapat kata dasarnnya saja.

\item Tahap terakhir pada text preprocessinf adalah menghitung bobot tiap kata menggunakan pembobotan TF-IDF dengan penggunaan kombinasi fitur 
unigram dan bigram.

\item Data kemudian dibagi 2, training data (80 persen) dan test data (20 persen). pemilihan training dan test data akan dilakukan secara acak oleh program.

\item Lakukan Klasifikasin teka menggunakan algoritma machine learing, seperti Multilayer Perceptron, Support Vector Machine, Naïve Bayes, dan Random Forest.

\item Bandingkan hasil antara algoritma berdasarkan tingkat akutasi, presisi, recall dan score F-1 dari tiap algoritma.
\end{enumerate}
\section{HASIL DAN PEMBAHASAN}
\subsection{Precision}
Nilai Precision adalah tingkat akurasi berdasarkan informasi yang diminta pengguna 
dengan jawaban yanh diberikan sistem.
Perhitungan akurasi adalah sebagai berikut (Manning et al, 2009):

$$ Precision=\frac{TP}{(TP + FP)}$$

dapat pula ditulis sebagai berikut:

$$ Precision=\frac{Relevant data found}{All data found}$$

Keterangan:

TP = True Positive adalah jumlah data relevan yang secara benar diklasifikasian sebagai kecocokan oleh sistem.

FP = False Positive adalah jumlah data yang tidak relevan, namjn diklasifikasikan sebagai kecocokan oleh sistem.

Adapun nilai Presisi yang didapat dari tiap algoritma dapat dilihat pada tabel dibawah

\begin{center}
Tabel 1 Nilai Precision

\begin{tabular}{ |p{5cm}||p{3cm}|  }
 \hline
 Algoritma & Precision \\
 \hline
 Multilayer Perceptron   & 69.12    \\
 Naïve Bayes &   79.79  \\
 Support Vector Machine & 70.16 \\
 Random Forest.    & 79.34 \\
 \hline
\end{tabular}
\end{center}
\subsection{Recall}
Recall adalah nilai kesuksesan sistem dalam menemukan kembali sebuah informasi, 
dengan kata lain Recall menunjukan selengkap apa hasil relevan yang ditampulkan 
sistem.

$$ Recall=\frac{TP}{(TP + FP)}$$

Keterangan:

TP = True Positive adalah jumlah data relevan yang secara benar diklasifikasian 
sebagai kecocokan oleh sistem.

FN = False Negative adalah jumlah data relevan, namun tidak diklasifikasikan sebagai 
kecocokan data oleh sistem.

Adapun nilai Recall yang didapat dari tiap algoritma dapat dilihat pada tabel dibawah

\begin{center}
Tabel 2 Nilai Recall

\begin{tabular}{ |p{5cm}||p{3cm}|  }
 \hline
 Algoritma & Precision \\
 \hline
 Multilayer Perceptron   & 67.16    \\
 Naïve Bayes &   66.86  \\
 Support Vector Machine & 62.31 \\
 Random Forest.    & 73.82 \\
 \hline
\end{tabular}
\end{center}

\subsection{ Skor F-1}
Skor F-1 adalah perhitungan nilai performa yang dilakukan untuk melihat hasil yabg 
didapat dari proses klasifikasi berdasarkan nilai Presisi dan recall yang telah didapat 
sebelumnya.

$$ F1 =\frac{2 \times recall \times precision}{recall + precision}$$

Perbandingan Skor F-1 dari keempat algortima dapat dilihat pada tabel 
berikut.

\begin{center}
Tabel 3 Nilai F-1

\begin{tabular}{ |p{5cm}||p{3cm}|  }
 \hline
 Algoritma & Precision \\
 \hline
 Multilayer Perceptron   & 67.09    \\
 Naïve Bayes &   67.26  \\
 Support Vector Machine & 71.08 \\
 Random Forest.  & 74.24 \\
 \hline
\end{tabular}
\end{center}

\subsection{Accuracy}
Level Accuracy adalah level kedekatan antara nilai prediksi dengan nilai aktual.
Perhitungan nilai akurasibdapat ditulis sebagai berikut(Manning et al, 2008):

$$ Accuracy =\frac{(TP + TN)}{TP + TN + FP + FN}$$

Perhitungan akurasi pada persamaan diatas juga dapat ditulis sebagai berikut :

$$ Accuracy =\frac{data yang diklasifikasikan dengan benar}{total data diuji}$$

Keterangan :

TP = True Positive adalah jumlah data relevan yang secara benar diklasifikasian 
sebagai kecocokan oleh sistem.

TN = True Negatif adalah jumlah data tidak relevan yang diklasifikasikan sebagai tidak 
cocok dengan benar oleh sisitem.

FP = False Positive adalah jumlah data yang tidak relevan, namjn diklasifikasikan 
sebagai kecocokan oleh sistem

FN = False Negative adalah jumlah data relevan, namun tidak diklasifikasikan sebagai 
kecocokan data oleh sistem.

Nilai Akurasi dari keemlat algoritma dapat dilihat pada tabel dibawah ini.

\begin{center}
Tabel 4 Nilai Accuracy

\begin{tabular}{ |p{5cm}||p{3cm}|  }
 \hline
 Algoritma & Precision \\
 \hline
 Multilayer Perceptron   & 68.63    \\
 Naïve Bayes &   74.51  \\
 Support Vector Machine & 60.00 \\
 Random Forest.    & 75.37 \\
 \hline
\end{tabular}
\end{center}

Secara keseluruhan perbandingan perbandingan antara keempat algoritma: Multilayer 
Perceptron, Support Vector Machine, Naïve Bayes, dan Random Forest berdasarkan 
presisi, recall, skor F-1 dan akurasi dapat dilihat pada tabel dibawah.

\begin{center}
Tabel 5 Perbandingan Algoritma

\begin{tabular}{ |p{5cm}||p{3cm}| |p{3cm}||p{3cm}||p{3cm}| }
 \hline
 Algoritma & P & R & F-1 & A \\
 \hline
 Multilayer Perceptron   & 69.12 & 67.16 & 67.09 & 68.63     \\
 Naïve Bayes &   79.79 & 66.86 & 67.26 & 74.51   \\
 Support Vector Machine & 70.16 & 62.31 & 71.08 & 60.00  \\
 Random Forest.    & 79.34 & 73.82 & 74.24 &   \\
 \hline
\end{tabular}
\end{center}

\section{KESIMPULAN}
Pada studi ini, sistem klasifikasi berita hoax di Indonesia telah dibuat menggunakan 
machine learing dengan 4 macam algoritma, antara lain : Multilayer Perceptron, Naïve
Bayes, Support Vector Machine, dan Random Forest. dengan total 150 artikel ssbagai 
data yang terdiri atas 50 artikel hoax dan 100 artikel non hoax. Penelitian ini juga 
dilakukan dengan melalui proses tokenizing, case folding, normalization, filtering, 
stopwords removing, stemming, and TF-IDF weighting menggunanakn fitur unigram
dan bigram.
Dapat disimpulkan bahwa hasil klasifikasi terbaik didapat dengan algoritma Random 
Forest jika dibandingkan dengan algoritma Multilayer Perceptron algorithm, Naïve 
Bayes, dan Support Vector Machine dengan akurasi sebesar 75.37%.
Berdasarkan perbandinga pada tabel 6, penulis lebih memilih untuk menggunakan 
algoritma dengan akurasi paling tinggi, karena secara rata-rata jika tingkat akurasinya 
tinggi, maka tingkat presisi dan recallnya juga akan tinggi pula

\begin{thebibliography}{99}
 \bibitem{DAFTAR PUSTAKA}{ Asiyah. S. N., Fithriasari. K., (2016): Klasifikasi Berita Online Menggunakan 
Metode Support Vector Machine dan K-Nearest Neighbor, Surabaya: Jurnal Sains 
dan Seni ITS.)} 
\bibitem{DAFTAR PUSTAKA}{Binarwati. L., Mukhlash. I., Soetrisno. S., (2017): Implementasi Algoritma 
Genetika untuk Optimalisasi Random Forest dalam Proses Klasifikasi Penerimaan 
Tenaga Kerja Baru :Studi Kasus PT.XYZ, Surabaya: Jurnal Sains dan Seni ITS.}
\bibitem{DAFTAR PUSTAKA}{] Juditha. C., (2018): Interaksi Simbolik dalam Komunitas Virtual Anti Hoak suntuk 
Mengurangi Penyebaran Hoaks, Jakarta: Jurnal PIKOM, vol. 19, no. 1, 
Kementerian Komunikasi dan Informatika RI.}
\bibitem{DAFTAR PUSTAKA}{Monohevita. L., (2017): Stop Menyebarkan Hoax, Depok: Universitas Indonesia.}
\bibitem{DAFTAR PUSTAKA}{ Negnevitsky. M., (2005): Artificial Intelligence: A Guide to Intelligent System (2nd 
Ed), Harlow: Pearson Education.}
\bibitem{DAFTAR PUSTAKA}{ Rasywir. E.,  Purwarianti. A., (2015): Eksperimen pada Sistem Klasifikasi Berita 
Hoax Berbahasa Indonesia Berbasis Pembelajaran Mesin. Bandung: Jurnal 
Cybermatika, vol. 3, no. 2.}
\end{thebibliography}

\end{document}